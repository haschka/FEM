\chapter{Basic Usage}

Within this chapter you will find all the necessary to install and use
this tool for your own problems. This tool implements the very basic
of Finite Element Theory, and only one specific element so
far. Nevertheless you might find it interesting to use if for your own
purposes. The author however encorages you to read at least the theory section
in order to know what you are actually doing. 

\section{Installation - Compiling the code}

\subsection{On GNU-Linux/BSD/Unix Systems}

The program has been programmed with portablitiy in mind. As this is
kind of an educatinal tool incorporating every thing one needs to make
a finite element calculation this tool does not require any external
library besides a standard C library.

In order to install the program one currently needs, make ( GNU make
4.1 has been tested ), a suiteable c compilier ( gcc 4.9.3 and clang
3.5.0 have been tested ), and c library ( glibc 2.21 has been
tested).

The supplied makefile can be adapted to suite compilier and necessary compilier
flags on your system and should work out of the box on common GNU/Linux
systems having the listed libraries installed.

Issuing make in the directory containing this program should suffice
to build the hexfem binary.

\subsection{On Windows}

Windows binaries can be compiled using the MingW64 toolchain.

\section{The hexfem input file}

Runnig this program, an input file is required. The input file is
modeled after the ANSYS input file syntax but has some significant
differences that are outlined below.
The input file has to be composed of the following statements:
\begin{itemize}
\item {\tt N, [node\_ID], [X], [Y], [Z]} \\
      Node definition: Defines a node in the the mesh with a
      corresponding unique identifier
      [node\_ID], at the coordinates [X], [Y], [Z].
      Each node can only to appear once in the input file. Double
      definitions where the last definition is taken into account as
      in ANSYS are not implemented and result in corrupt results. A
      node has to be defined before
      force or constraint definitions acting on that node. 
\item {\tt Mat, [material\_ID]} \\
      Material pointer: points to a material with a unique id. Each
      material pointer can only be defined once. Double usage of the
      same statements is per se not allowed in the input file. The
      latest material pointer in front of an
      element specification is used in order to attribute the
      corresponding material to that element.
\item {\tt MP, EX, [material\_ID], [value]} \\
      Young's modulus: Defines the Youngs modulus of the material
      indicated by the [material\_ID] statement. A young's modulus
      can be located anywhere in the input file. Materials are so far
      limited to an istropic Youngs modulus, and hence the EX field
      is valid for all directions.
\item {\tt MP, NUXY, [material\_ID], [value]} \\
      Poisson's ratio: Defines Poisson's ratio of the
      material. Such statement can
      just as the Young's modulus be defined anywhere within the input
      file. Due to isotropicity
      the NUXY field defines the Poisson's ratio for all directions.
\item {\tt E, [N], [N], [N], [N], [N], [N], [N], [N]} \\
      Element definition: Defines a hexahedral element using the nodes
      [N], where [N] is the [node\_ID] of a node defined in the input
      file. Each element 
      specification obtains the material indicated by a Material
      pointer in front of it. Otherwise these statements can be placed anywhere
      in the input file.
\item {\tt D, [node\_ID], ALL, 0}
      Displacement constraint: Defines a constraint on a
      node. Currently only removing all degrees of freedom and hence,
      fixing a nodes position is implemented. The statement
      requires the node with [node\_ID] to be already defined.
\item {\tt F, [node\_ID], [Direction], [value]}
      Force definition: Defines the external force to be applied on a
      node.
      The [Direction] field can be any of FX, FY, FZ in order to specify specfic
      components. This statement can only occur after the node with
      [node\_ID] has been specified in the input file.
\item {\tt ZOU, [specification]}
      Output specification: Defines the values shall be calculated
      and printed.
      The [specification] field can currently be any of: DIS for nodal
      displacemnts, FOR for forces exercised on nodes, STE for an
      output of a strain tensor for each element, PST for an output of
      principal strains for each element, VOB for the volumes of each
      element before deformation and VOA for the volumes of each
      element after deformation. VOA, VOB, STE and PST are requiring
      additional computational time, while DIS and FOR are always calculated,
      but not printed if the statements are missing.
      
      
\end{itemize}

A sample input file is shown in listing \ref{lst-input-file}.
\lstset{language=,caption={Input file containing a brick made up of two
    elements that is locked at its base and under load from the
    top},label=lst-input-file}
\begin{lstlisting}
N, 1, 0, 0, 0
N, 2, 1, 0, 0
N, 3, 1, 1, 0
N, 4, 0, 1, 0
N, 5, 0, 0, 1
N, 6, 1, 0, 1
N, 7, 1, 1, 1
N, 8, 0, 1, 1
N, 9, 0, 0, 2
N, 10, 1, 0, 2
N, 11, 1, 1, 2
N, 12, 0, 1, 2

D, 1, ALL, 0
D, 2, ALL, 0
D, 3, ALL, 0
D, 4, ALL, 0

F, 9, FZ, -5.
F, 10, FZ, -5.
F, 11, FZ, -5.
F, 12, FZ, -5.

MAT, 1
MP, EX, 1, 300.
MP, NUXY, 1, 0.4

E, 1, 2, 3, 4, 5, 6, 7, 8
E, 5, 6, 7, 8, 9, 10, 11, 12

ZOU, DIS
ZOU, FOR

ZOU, STE
ZOU, PST

ZOU, VOB
ZOU, VOA
\end{lstlisting}

\section{Running the Program/Program Ouput}

Successful compilation of the program results in a binary file
named hexfem.  This file can then be run using a correctly constructed
input file. It will print the requested outputs to the standard output
on the terminal.

A command like the following may be invoked to execute the program:

\lstset{language=bash,caption={Running the finite element code},label=lst-run}
\begin{lstlisting}
./hexfem [Inputfile]
\end{lstlisting}
Here [Inputfile] is the input file in the syntax specified.

As the output is directly printed to \emph{standard out},
it is suggested to redirect the standard output to a file
using the capabilities that the different operating system shells
provide.
