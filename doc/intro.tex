\chapter{Introduction}

Welcome to the Manual of the Hexaheadral Finite Element Solver (hexfem).
The program implements a hexahedral element with incompatible displacement
modes \cite{incomp1, incomp2}. The hexahedral element implemented is
strictly ressembling the ANSYS \emph{Solid 45} element. This program
shall serve both as an educational tool to teach about the concepts of
a finite element solver as well as its numerical implemetation. As
such the code was designed to be short and efficent. It does not use
any external libraries besides the C library. It implements the finite
elements, a sparse matrix system, and a sparse matrix conjugate
gradient with jabocbi preconditioning. The solver is further available
in vectorized forms for the SSSE3 and AVX2 instruction sets found on
computers using the X86\_64 instruction set. \newline \newline
The following manual has three sections:
\newline \newline
\begin{enumerate}
\item{Basic Usage: All the necessary to use this finite element
  solver, without thinking to much about it.}
\item{Theory: The theory that makes such a tool possible. This shall
  not serve as a finite element course replacement but rather as a
  summery of what is implemented herin, and provide an overview of
  what is necessary to get such a tool baked together.}
\item{Code Algorithms: Some hints for those who want to mess with the
  code of this tool, expand it, modify it for their own purpose.}
\end{enumerate}
\newpage
The license of the software:\\ \\
(c) 2016 Thomas Haschka\\ \\
This software is provided 'as-is', without any express or implied
warranty. In no event will the authors be held liable for any damages
arising from the use of this software. \\
Permission is granted to anyone to use this software for any purpose,
including commercial applications, and to alter it and redistribute it
freely, subject to the following restrictions: \newline
The origin of this software must not be misrepresented; you must not
claim that you wrote the original software. If you use this software
in a product, an acknowledgment in the product documentation would be
appreciated but is not required.\newline
Altered source versions must be plainly marked as such, and must not
be misrepresented as being the original software. \newline
This notice may not be removed or altered from any source distribution.
