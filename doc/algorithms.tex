\chapter{Theory}

\section{Static Finite Element Analyses}

This hexaheardral finite element solver implements the so called \emph{direct
  stiffness appoach}
\cite{bathe}. At the center of this approach lies the solution of the
linear sparse system:
\begin{equation}
  K_{ij} u_j = F_i
  \label{eqn-force-displacement}
\end{equation}
where $K$ is the stiffness matrix, $u$ the displacement vector and $F$
the vector of applied forces. For $n$ nodes $i$ and $j$ ranges from
$1 \le i,j \le 3n$  as each
node has three degrees of freedom.
The problem can thus be solved in finding the
appropriate stiffness matrix $K$ for the mesh in question. 
Solve the linear system (\ref{eqn-force-displacement}) one obtains the
displacements of each node. These displacements allow us then
calculate strains, stresses as well as other linked results with relative
ease.

\subsection{Determination of the Stiffness Matrix}

As we sketch the algorithm, we start with the description of how we derive
the stiffness matrix for a system. The basic idea here is that one can
determine a local stiffness matrix for each finite element in the
system. Once such a local stiffness matrix is obtained, it can, due to
linearty, be summed into the global stiffness matrix.
The dertermination of the local stiffness matrix 
requires several fine tricks. The first such trick is to transpose the element
in question into an easy to integratable local coordiante system ${\xi,
\eta,\zeta}$ where each element represents a perfect cubic shape with
an edgelength of 2, ranging from $-1 \le \xi,\eta,\zeta \le 1$. The
coordinate transformations for the element are then described by so
called shape functions. As we only treat hexaheadral elements with
this solver the shape functions are given by the following equations:

\begin{eqnarray}
  h_1 &=& \frac{1}{8}(1-\xi)(1-\eta)(1-\zeta), \nonumber \\
  h_2 &=& \frac{1}{8}(1+\xi)(1-\eta)(1-\zeta), \nonumber \\
  h_3 &=& \frac{1}{8}(1+\xi)(1+\eta)(1-\zeta), \nonumber \\
  h_4 &=& \frac{1}{8}(1-\xi)(1+\eta)(1-\zeta), \nonumber \\
  h_5 &=& \frac{1}{8}(1-\xi)(1-\eta)(1+\zeta), \nonumber \\
  h_6 &=& \frac{1}{8}(1+\xi)(1-\eta)(1+\zeta), \nonumber \\
  h_7 &=& \frac{1}{8}(1+\xi)(1+\eta)(1+\zeta), \nonumber \\
  h_8 &=& \frac{1}{8}(1-\xi)(1+\eta)(1+\zeta),
  \label{eqn-shape}
\end{eqnarray}
together with the coordinate transformation:
\begin{equation}
  x = \sum_{i=1}^8x_ih_i,\qquad y = \sum_{i=1}^8y_ih_i, \qquad z =
  \sum_{i=1}^8z_ih_i,
  \label{eqn-cotrans}
\end{equation}
where $x_i$,$y_i$,$z_i$ are the coordinates of the hexheadral element's
nodes in the global reference frame.  

So called non conforming displacement nodes, introduce three further
functions which are of special interest:
\begin{equation}
  p_1 = (1-\xi^2),\qquad 
  p_2 = (1-\eta^2),\qquad 
  p_3 = (1-\zeta^2).
  \label{eqn-bubble}
\end{equation}
These functions are not coordinate transformations in the sens that they are not
reltated to any node. They can be interpreted as a deformation of the element in
a parabobla-like way along the $\xi$,$\eta$ or $\zeta$ axes. Therefore, they are
further known by the name \emph{bubble functions} in literature.
These functions are necessary to avoid the so called \emph{shear locking}
phenomena in the mesh \cite{incomp1}.

Th shape functions allow us to evaluate the Jacobian $J$ and
its inverse $J^{-1}$:
\begin{equation}
  J = \frac{\partial(x,y,z)}{\partial(\xi,\eta,\zeta)},\qquad
    J^{-1} = \frac{\partial(\xi,\eta,\zeta)}{\partial(x,y,z)},
    \label{jacobi}
\end{equation}
which, looking at the edges of the elements (defined by the nodes)
where both $x$,$y$,$z$ and $\xi$,$\eta$,$\zeta$ are known, is found to be:
\begin{equation}
  J = \left[
    \begin{array}{ccc}
      \sum_{i=1}^8x_i\frac{\partial h_i}{\partial\xi} &
      \sum_{i=1}^8y_i\frac{\partial h_i}{\partial\xi} &
      \sum_{i=1}^8z_i\frac{\partial h_i}{\partial\xi} \\
      \sum_{i=1}^8x_i\frac{\partial h_i}{\partial\eta} &
      \sum_{i=1}^8y_i\frac{\partial h_i}{\partial\eta} &
      \sum_{i=1}^8z_i\frac{\partial h_i}{\partial\eta} \\
      \sum_{i=1}^8x_i\frac{\partial h_i}{\partial\zeta} &
      \sum_{i=1}^8y_i\frac{\partial h_i}{\partial\zeta} &
      \sum_{i=1}^8z_i\frac{\partial h_i}{\partial\zeta}
    \end{array}
    \right].
  \label{eqn-jacobien}
\end{equation}

Using d'Alemberts variational calculus one has the possibility to show from
physical principals that \cite{bathe}:
\begin{equation}
  K = \int_VB^{T}DB dV,
  \label{eqn-stiffness}
\end{equation}
where V is the volume occupied, $B$ is the strain displacement
relation which we will be describe in the following. $D$ is the stress
strain relation, which can be derived from Hooke's law
\cite{mechanics} in three dimensions and is given by: 
\begin{eqnarray}
  \left[\begin{array}{c}
      \sigma_{11} \\ \sigma_{22} \\ \sigma_{33} \\ \sigma_{12} \\
      \sigma_{23} \\ \sigma_{13}
    \end{array}
    \right]
  &=& \Gamma\left[\begin{array}{cccccc}
    1-\nu & \nu & \nu & 0 & 0 & 0 \\
    \nu & 1-\nu & \nu & 0 & 0 & 0 \\
    \nu & \nu & 1-\nu & 0 & 0 & 0 \\
    0 & 0 & 0 & \frac{1}{2}(1-2\nu) & 0 & 0 \\
    0 & 0 & 0 & 0 & \frac{1}{2}(1-2\nu) & 0 \\
    0 & 0 & 0 & 0 & 0 & \frac{1}{2}(1-2\nu)
    \end{array}
    \right] 
  \left[\begin{array}{c}
      \epsilon_{11} \\ \epsilon_{22} \\ \epsilon_{33} \\ 2\epsilon_{13} \\
      2\epsilon_{23} \\ 2\epsilon_{13}
    \end{array}
    \right] \nonumber \\
  &=& D \left[\begin{array}{c}
      \epsilon_{11} \\ \epsilon_{22} \\ \epsilon_{33} \\ 2\epsilon_{12} \\
      2\epsilon_{23} \\ 2\epsilon_{13}
    \end{array}
    \right],
\end{eqnarray}
with the prefactor $\Gamma = \frac{E}{(1+\nu)(1-2\nu)}$. Here $E$ is
the Young's modulus and $\nu$ the Poisson material constant.
The strain displacement relation can be obtained from the partial
derivatives of the shape function:
\begin{equation}
  \left[\begin{array}{c}
      \epsilon_{11} \\ \epsilon_{22} \\ \epsilon_{33} \\ 2\epsilon_{23} \\
      2\epsilon_{13} \\ 2\epsilon_{12}
    \end{array}
    \right] = \left[\begin{array}{ccc}
      \frac{\partial h_i}{\partial x} & 0 & 0 \\
      0 & \frac{\partial h_i}{\partial y} & 0 \\
      0 & 0 & \frac{\partial h_i}{\partial z} \\
      \frac{\partial h_i}{\partial y} &
      \frac{\partial h_i}{\partial x} & 0 \\
      \frac{\partial h_i}{\partial z} & 0 &
      \frac{\partial h_i}{\partial y} \\
      \frac{\partial h_i}{\partial z} & 0 & \frac{\partial h_i}{\partial x}
    \end{array}
    \right] \left[\begin{array}{c} u_{i,x} \\ u_{i,y}
      \\ u_{i,z} \end{array}\right] =
  B
  \left[\begin{array}{c} u_{i,x} \\ u_{i,y}
      \\ u_{i,z} \end{array}\right].
  \label{eqn-strain-displacement}
\end{equation}
In (\ref{eqn-strain-displacement}) $i$ includes all the shape functions
going from 1 to 8, which
creates a repeated pattern in the matrix. Hence, for a single
hexahedral element $B$ is a 24 by 6 matrix. In the same way $u_{i,xyz}$
is a 24 component vector. The operators needed in
order to evaluate the partial derivatives in equation
(\ref{eqn-strain-displacement}) can be obtained by analytically
deriving the shape functions shown in (\ref{eqn-shape}) 
and in calculating the inverse of the Jacobian matrix (\ref{eqn-jacobien}):
\begin{equation}
  \left[\begin{array}{c}
      \frac{\partial p_i}{\partial x} \\
      \frac{\partial p_i}{\partial y} \\
      \frac{\partial p_i}{\partial z}
    \end{array}\right]
  =
  J^{-1}{(\xi,\eta,\zeta)}
  \left[\begin{array}{c}
      \frac{\partial p_i}{\partial \xi} \\
      \frac{\partial p_i}{\partial \eta} \\
      \frac{\partial p_i}{\partial \zeta}
    \end{array}\right]
\end{equation}

Writing down these
matricies we did not account for the supplimentary bubble functions
in equation (\ref{eqn-bubble}).

As the bubble functions are not connected to any nodes, a trick called
static condensation is used in order for them to be affective. This 
allows us to take the virtual displacement coordinates $a$ and the
virtual displacement matrix $P$ into account. Where $a$ is the virtual
analogue to displacements $u$ and the matrix $G$ to the strain-displacement
relation $B$. As there are three bubble functions (\ref{eqn-bubble})
three further virtual displacement coordinates $a$ are introduced,
resulting in 9 virtual degrees of freedom. For a single element $G$ is
hence, a 9 by 6 matrix, taking for three components the same form as $B$.
To clarify this let us rewrite (\ref{eqn-stiffness}) taking these
virtual displacents into account: 

\begin{equation}
  K = \left[\begin{array}{cc}
      \int_V B^{T}DB dV & \int_V B^{T}DG dV \\
      \int_V(B^{T}DG)^{T} dV & \int_VG^{T}DG dV
    \end{array}\right],
  \label{eqn-stiffness-and-virtual}
\end{equation}

with,

\begin{equation}
  G = \left[\begin{array}{ccc}
      \frac{\partial p_i}{\partial x} & 0 & 0 \\
      0 & \frac{\partial p_i}{\partial y} & 0 \\
      0 & 0 & \frac{\partial p_i}{\partial z} \\
      \frac{\partial p_i}{\partial y} &
      \frac{\partial p_i}{\partial x} & 0 \\
      \frac{\partial p_i}{\partial z} & 0 &
      \frac{\partial p_i}{\partial y} \\
      \frac{\partial p_i}{\partial z} & 0 & \frac{\partial h_i}{\partial x}
    \end{array}
    \right].
\end{equation}

Further as pointed out by R. L. Taylor et al. in \cite{incomp2} in
order for the elements to pass the patch test $G$ should not influence
the volume of the element. The calculation of $G$ requires just as the
calculation of $B$ terms that result from the inverse Jacobian
\ref{eqn-jacobien}. The trick here is to take the Jacobien at the
center of the element, at $\xi=\eta=\zeta=0$ calculating
$\frac{\partial p_i}{\partial x}$, $\frac{\partial p_i}{\partial y}$
and $\frac{\partial p_i}{\partial z}$.
Symbolically written down to be:
\begin{equation}
  \left[\begin{array}{c}
      \frac{\partial p_i}{\partial x} \\
      \frac{\partial p_i}{\partial y} \\
      \frac{\partial p_i}{\partial z}
    \end{array}\right]
  =
  J^{-1}|_{\xi=\eta=\zeta=0}
  \left[\begin{array}{c}
      \frac{\partial p_i}{\partial \xi} \\
      \frac{\partial p_i}{\partial \eta} \\
      \frac{\partial p_i}{\partial \zeta}
    \end{array}\right]
  \frac{|J(\xi=\eta=\zeta=0)|}{|J(\xi,\eta,\zeta)|},
  \label{eqn-j-null}
\end{equation}
with the correction factor $\frac{|J(\xi=\eta=\zeta=0)|}{|J(\xi,\eta,\zeta)|}$.

The bubble functions $p$ shall in no further way influence the coordinate
transformation and hence, the jacobian is
always calculated by use of the classical shape functions
(\ref{eqn-shape}). The jacobian is always varied when used
in evaluating $B$ or as a determinant in the volume element $dV$ of
equation. It is only kept constant $\xi=\eta=\zeta=0$ at calculating the partial
derivatives:
$\frac{\partial p_i}{\partial x}$, $\frac{\partial p_i}{\partial y}$ and
$\frac{\partial p_i}{\partial z}$ for equation  
(\ref{eqn-stiffness-and-virtual}) using (\ref{eqn-j-null}).

Static condensation now consists of matrix coefficient
modification. Looking at equation (\ref{eqn-stiffness-and-virtual})
the trick here is to turn the lower part into the identity
matrix. Hence, the system takes the form:
\begin{equation}
  K^* = \left[\begin{array}{cc}
      \left[\int_V(B^{T}D B)dV\right]_{G-inc} & 0 \\
      0 & I
    \end{array}\right],
  \label{eqn-static-cond}
\end{equation}
where using partial guassian elemination the lower part is transfomed
into the identity matrix $I$ modifying the coefficients in the
$\int_VB^{T}DB dV$ part of the matrix, incorporating the virtual stress-strain
terms in the matrix \cite{static-cond}. Looking at equation $K_{ij} u_j = F_i$
(\ref{eqn-force-displacement}) this is possible as we are not
interested in virtual displacements $a$ that as the force acting on
this points can only be $0$ as we only let the force act on real
nodes. $K^* = \left[\int_V(B^{T}D B)dV\right]_{G-inc}$ for local
elementary stiffness matrices can then due to linearity directly
summed into the global stiffness matrix at positions effecting the
correct nodal displacemnts $u$.

The static condensation algorithm further also impacts the strain-displacement
relation, $B$. E. L. Wilson noted this \cite{static-cond} and provided
an algorithm in order to manipluate the stress-displacement relation
denoted in the paper as $A$. In our tests the implementation shown
however did not result in the same results as ANSYS would print
out. We hence, based ourselfs on a different method noting that the
static condensation can be seen expressed as the application of a
matrix $M$ in such that: 
\begin{eqnarray}
  K^* = MK &=& M \left[\begin{array}{cc}
      \int_V B^{T}DB dV & \int_V B^{T}DG dV \\
      \int_V(B^{T}DG)^{T} dV & \int_VG^{T}DG dV
    \end{array}\right] \nonumber \\ 
     &=& \left[\begin{array}{cc}
      \left[\int_V(B^{T}D B)dV\right]_{G-inc} & 0 \\
      0 & I
    \end{array}\right] , \label{eqn-partial-inverse}
\end{eqnarray}
as we can see from equations
(\ref{eqn-stiffness-and-virtual},\ref{eqn-static-cond}). M is thus a
partial inverse of the matrix $K$ which can be obtained by
the traditional gauss jordan elemination method with an extended 
identity matrix. The only difference is here that the elemination is
stopped once the lower left section of $K$ given by $\int_VG^{T}DG dV$
is transformed into the identity matrix.

Knowing $M$, the condensated $B$ used for further stress $\epsilon$ or
strain $\sigma$. These evalutions can be obtained using the following
formula:
\begin{equation}
  B^T_{G-inc}= M[B G]^T,
\end{equation}
where $[B G]$ discribes the a 33 by six matrix which contains $B$
extended by $G$ components. Matrix multiplaction can be stopped so
that $B_{G-inc}$ results in a 24 by six matrix. The static condensation
\cite{static-cond} paper evokes that this step could be done without
the matrix multiplication, and by deducing the values directly during
the Gauss Jordan elemination process. As $[B G]$ changes for every
point in an element $M$ stays static. Performing the reduction during
the elemination process would have meant to change the eight $B$
matrices for one element at the same time. We believe that it is more
advantageous and intuitive to store this information once in matrix
$M$.

\subsection{Integration, Volume and Mean}

Integration in all our equations is effectuated using a gaussian
quadrature approximation, in the $\xi$, $\eta$, $\zeta$ coordinate
system. Integration over an element is thus numerically performed in the
follow way: 
\begin{eqnarray}
  \int_V A(x,y,z) dV &=& \int_{-1}^{1}\int_{-1}^{1}\int_{-1}^{1} A(\xi,\eta,\zeta)
  |J| d\xi d\eta d\zeta \nonumber \\
  &\approx& \sum_{i=1}^8 A(r_i) |J(r_i)|,
\end{eqnarray}
where $|J|$ is the determinant of the Jacobian. $A$ is scalar, vector
or tensor field to be integrated and $r_i$ are the gaussian quadrature
points found to be:
\begin{eqnarray}
  r_1&=& \frac{1}{\sqrt{3}}[-1,-1,-1]^T \nonumber \\
  r_2&=& \frac{1}{\sqrt{3}}[1,-1,-1]^T \nonumber \\
  r_3&=& \frac{1}{\sqrt{3}}[1,1,-1]^T \nonumber \\
  r_4&=& \frac{1}{\sqrt{3}}[-1,1,-1]^T \nonumber \\
  r_5&=& \frac{1}{\sqrt{3}}[-1,-1,1]^T \nonumber \\
  r_6&=& \frac{1}{\sqrt{3}}[1,-1,1]^T \nonumber \\
  r_7&=& \frac{1}{\sqrt{3}}[1,1,1]^T \nonumber \\
  r_8&=& \frac{1}{\sqrt{3}}[-1,1,1]^T.
\end{eqnarray}
in $\xi$, $\eta$, $\zeta$ coordinates. Using this numerical approximation
the volume of a hexahedral cell casts into:
\begin{equation}
  V = \sum_{i=1}^8 |J(r_i)|
\end{equation}
In order to obtain numerical compatibility stress and strain are
values are currently exempt from this procedure and are simply
meaned without taking the volume into account and hence, are given by:
\begin{equation}
  \bar{\epsilon} = \frac{1}{8}\sum_{i=1}^8\epsilon(r_i).
\end{equation}

\subsection{Solution of a Linear System}

The system to be solved is in general a linear sparse symmetric
system, as highlighted by equation (\ref{eqn-force-displacement}).

As a linear, operator in our case the $n$ by $n$ stiffness matrix $K$, forms an
inner product. Hence, one can define a base of $n$ dimensional space
associated with this inner product. Let us denote:
\begin{equation}
  {p_1,p_2,\ldots,p_n},
\end{equation}
a set of orthonormal vectors in respect to the inner product:
\begin{equation}
  \left<u|v\right>_K = u^{T}Kv
\end{equation}
to be a base of our $n$ dimensional space. We can then develop a
solution for our problem $Ku=F$, in this base:
\begin{equation}
u = \sum_{i=1}^n\alpha_ip_i
\end{equation}
and further:
\begin{eqnarray}
  Ku&=&\sum_{i=1}^n\alpha_i Kp_i \\
  p^{T}Ku&=&\sum_{i=1}^N\alpha_ip_k^{T}Kp_i \\
  p_k^{T}F&=&\sum_{i=1}^N\alpha_i\left<p_k|p_i\right>_K \\
\end{eqnarray}
from where we obtain, the coefficients:
\begin{equation}
  \alpha_k=\frac{p_k^{T}F}{p_k^{T}Kp_k}
\end{equation}
which means that a solution for $u$ can be found in finding a set of
orthonormal (conjugate) vectors according to our $n$ dimensional space
with the inner product defind by the stiffness matrix, and further
calculating the coefficients $\alpha_k$.

Using this as an iterative method in our case we define a residual at
step $k$ to be:
\begin{equation}
  r_k = F - Ku_k.
\end{equation}
From where, like at Gram Schmitt Orthonormalisation we want to find a
vector normal on the previous vectors:
\begin{equation}
p_k = r_k - \sum_{i<k}\frac{p_i^TKr_k}{p_i^TKp_i}p_i.
\end{equation}
The improved solution is then given by:
\begin{equation}
  u_{k+1} = u_{k}+\alpha_kp_k.
\end{equation}
This works as due to the normalisation step $p_k$ is orthogonal to
$u_{k-1}$, according to the inner product defined by $K$.

As $r_{k+1}$ is in fact orthogonal to all $p_i$ the sum is not needed
and the algorithm can be very efficiently implemented by only knowing
previous $r_k$,$p_k$, and $u_k$. We initialize with:
\begin{eqnarray}
  r_0&=&F - Ku_0, \\
  p_0&=&r_0, \\
\end{eqnarray}
use the recurrence relations:
\begin{eqnarray}
  \alpha_k&=&\frac{r_k^{T}r_k}{p_k^{T}Kp_k}, \label{eqn-rec-1} \\
  u_{k+1}&=&u_{k}+\alpha_kp_k, \label{eqn-rec-2} \\
  r_{k+1}&=&r_k - \alpha_kKp_k, \label{eqn-rec-3} \\
  \beta_k&=&\frac{r_{k+1}^Tr_{k+1}}{r_k^Tr_k}, \label{eqn-rec-4} \\
  p_{k+1}&=&r_{k+1}+\beta_kp_{k} \label{eqn-rec-5},
\end{eqnarray}
and repeat until $r_k$ is sufficiently small. The convergence of the
algorithm outlined here largely depends on the properties of the
system to be solved. In order to speed up convergence a so called
preconditionning matrix $M$ can be used which results in the solution
of the equivalent linear system:
\begin{eqnarray}
  E^{-1}K(E^{-1})^T\hat{u}&=&E^{-1}F, \\
  EE^{T}&=&M, \\
  \hat{u}&=&E^Tu.
\end{eqnarray}
This changes the algorithm outlined above to an initialisation using:
\begin{eqnarray}
  r_0&=&F - Ku_0, \\
  z_0&=&M^{-1}r_0 \\
  p_0&=&z_0, \\
\end{eqnarray}
and the recurrence relations to:
\begin{eqnarray}
  \alpha_k&=&\frac{r_k^{T}z_k}{p_k^{T}Kp_k}, \label{eqn-recm-1} \\
  u_{k+1}&=&u_{k}+\alpha_kp_k, \label{eqn-recm-2} \\
  r_{k+1}&=&r_k - \alpha_kKp_k, \label{eqn-recm-3} \\
  z_{k+1}&=&M^{-1}r_{k+1}, \label{eqn-recm-4} \\
  \beta_k&=&\frac{z_{k+1}^Tr_{k+1}}{z_k^Tr_k}, \label{eqn-recm-5} \\
  p_{k+1}&=&z_{k+1}+\beta_kp_{k} \label{eqn-recm-6},
\end{eqnarray}
The preconditioning matrix $M$ can be more or less complicated. In
the optimal case $M^{-1}$ would be the inverse of $K$. In
our implemented algorithm we used the so called Jacobi precontionning
method setting $M$ to the diagonal of $K$, and 0 elsewhere. Even
though this method is almoast trivial and straightforward to
implement it is suggested to be one of the fasted methods on SIMD (vector)
systems \cite{precond}.

